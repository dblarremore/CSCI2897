\documentclass[11pt,onecolumn,superscriptaddress,notitlepage]{article}

\usepackage[total={6.5in,9in}, top=1.0in, includefoot]{geometry}
\usepackage{epsfig}
\usepackage{subfigure}
\usepackage{placeins}
\usepackage{amsmath}
\usepackage[usenames,dvipsnames,svgnames,table]{xcolor}
\usepackage{amssymb}
\usepackage{setspace}
\usepackage{graphicx} % Include figure files
\usepackage{times}
\usepackage{amsthm}
\usepackage{hyperref}
\usepackage[affil-it]{authblk} 
\hypersetup{bookmarks=true, unicode=false, pdftoolbar=true, pdfmenubar=true, pdffitwindow=false, pdfstartview={FitH}, pdfcreator={Daniel Larremore}, pdfproducer={Daniel Larremore}, pdfkeywords={} {} {}, pdfnewwindow=true, colorlinks=true, linkcolor=red, citecolor=Green, filecolor=magenta, urlcolor=cyan,}

\usepackage{enumitem}

\newcommand{\dx}[0]{\displaystyle\frac{d}{dx}}

\usepackage{parskip}

\date{}
\begin{document}

%%%%%%%%%% Authors
\author{CSCI 2897 - Calculating Biological Quantities - Larremore - Spring 2021}
%%%%%%%%%% Title
\title{Homework 1}
%%%%%%%%%% Abstract
\maketitle
%%%%%%%%%% Content

    %%%    
    %%%   
  %%%%%%%
   %%%%%
    %%%
     %
{\bf Notes:} Remember to (1) familiarize yourself with the collaboration policies posted on the Syllabus, and (2) turn in your homework to Canvas as a {\bf single PDF}. Hand-writing some or most of your solutions is fine, but be sure to scan and PDF everything into a single document. Unsure how? Ask on Slack! 

\section*{Pushups}

{\bf Calculate these derivatives.}

\begin{enumerate}
	\item $\dx x^2 = $
	\item $\dx x^{-2} = $
	\item $\dx e^{\pi x} = $
	\item $\dx e^{\pi x^{-2}} = $
	\item $\dx \ln{x} = $
\end{enumerate}

\section*{Squats}

{\bf Find solutions to each of these differential equations.}\footnote{Hint: ask yourself, ``What function, if I were to take its derivative, would satisfy this equation?''}

\begin{enumerate}[resume]
	\item $\displaystyle\frac{dy(t)}{dt} = 0$
	\item $\displaystyle\frac{dy(t)}{dt} = t$
	\item $\displaystyle\frac{dy(t)}{dt} = y(t)$
\end{enumerate}

\section*{Cooldown}
You may remember that there are infinitely many solutions to so-called {\it indefinite integrals} (integrals without limits of integration). For example, $$\int x \ dx = x^{2}/2 + c,$$ where $c$ is an arbitrary constant. If we take a derivative of both sides of that equation, we see that indeed, $x = \frac{d}{dx} \left(x^2/2+c\right)$, no matter what $c$ is.
\begin{enumerate}[resume]
	\item When solving the equations in the Squats section above, did arbitrary constants factor into your solutions? If not, why not? If so, how?
	\item In a couple sentences, in your own words, can you explain what it means for a differential equation to have a {\it family of solutions}?  
\end{enumerate}

\clearpage
\section*{Going viral}

A new semester is starting, and, feeling a sense of renewal, you log into your TikTok app to create some Day 1 content for your followers. Whatever you do, it works, and your content gets reblogged or retweeted or whatever the appropriate terms is for TikTok a great many times, and it hits you that you are going viral.  

Naturally, you are enrolled in {\it Calculating Biological Quantities}, and seeing loads of notifications pouring in, you consider the creation of a dynamical model to get a better understanding of how many {\it unread notifications} you are likely to see in the future. You make the following observations in your Influencer Lab Notebook. First, you record your observations every 10 minutes. Each time you record your observations, you address 20\% of the unread notifications. Second, you notice that {\it new} notifications are arriving at a growing rate: 10 in the first 10 minutes, 20 in the second 10 minutes, 30 in the next 10 minutes, 40, 50, $\dots$. Goodness Gracious! 

\begin{enumerate}[resume]
	\item For this model, would you consider continuous or discrete time? Explain. 
	\item What is the variable of this model? Jot down two questions about the variable that you could use your model to answer. 
	\item What would be an appropriate timescale? 
	\item Draw a Life Cycle Diagram for this model. 
	\item Use the Life Cycle Diagram to write down (a) a recursion equation and (b) a difference equation.
	\item Using Python, {\it code up your model}. How many unread notifications will there be after a week, assuming that you start with zero unread notifications? (You may assume that you have hired a social media manager who deals with your inbox, exactly like you would, while you sleep. Or assume that you don't sleep.)
	\item Produce a plot of unread notifications vs time during that week. Be sure to label your axes and include a legend. 
\end{enumerate}

{\it Extra Credit:} To consider more generic scenarios of going viral, identify parameters in the model above and replace them with letters. What are the units of those parameters? Produce a plot of the first week of unread notifications using different parameters, and write 2-3 sentences describing how the parameters affected the plot, as compared with the baseline scenario. 

\clearpage
\section*{Trust but verify}

\begin{enumerate}[resume]
	\item What is the order of this equation? Is it linear or nonlinear? 
	$$(1-x)y'' - 4xy' + 5y = cos(x)$$
	\item What is the order of this equation? Is it linear or nonlinear? 
	$$(\sin \theta) y''' - (\cos \theta) y' = 2$$
	\item Verify that the function solves the ODE or show that it does not. 
	$$2y' + y = 0; \qquad y=e^{-x/2}$$
	\item Verify that the function solves the ODE or show that it does not.\footnote{Gross! Refresh yourself on the product rule, the chain rule, and the Review of Differentiation on the inner cover of Zill.}
	$$y'' + y = \tan{x}; \qquad y=(\cos{x})\ln[\sec{x} + \tan{x}]$$
	\item Verify that the function solves the ODE or show that it does not.
	$$y'' + y = \tan{x}; \qquad y=-(\cos{x})\ln[\sec{x} + \tan{x}]$$
	\item Find a value of the parameter $a$ so that the function $y(t)=e^{at}$ is a solution to the given differential equation.
	$$ 5y' = 2y$$
	\item Find a value of the parameter $r$ so that the function $n(t) = a r^t$ is a solution to the given recursion. Comment on what you can say about the parameter $a$. 
	$$ n(t+1) = n(t) + n(t-1)$$
\end{enumerate}

\section*{Last but not least}
\begin{enumerate}[resume]
	\item Describe a time-varying phenomenon in the biological or physical world around you that makes you curious. Tell me a bit about it, pose a quantitative question about it, and write down the processes or steps in a model with a corresponding diagram (of any of the types we have discussed in class). Identify parameters and their units, choose a timescale, and if possible, write some equations for your model. This is a {\it creative} exercise so you will not be evaluated on whether the model is correct or plausible. No need to solve or analyze the equations. 
\end{enumerate}


     %
    %%%
   %%%%%
  %%%%%%%
    %%%
    %%%

\end{document}