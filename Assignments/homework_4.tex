\documentclass[11pt,onecolumn,superscriptaddress,notitlepage]{article}

\usepackage[total={6.5in,9in}, top=1.0in, includefoot]{geometry}
\usepackage{epsfig}
\usepackage{subfigure}
\usepackage{placeins}
\usepackage{amsmath}
\usepackage[usenames,dvipsnames,svgnames,table]{xcolor}
\usepackage{amssymb}
\usepackage{setspace}
\usepackage{graphicx} % Include figure files
\usepackage{times}
\usepackage{amsthm}
\usepackage{hyperref}
\usepackage[affil-it]{authblk} 
\hypersetup{bookmarks=true, unicode=false, pdftoolbar=true, pdfmenubar=true, pdffitwindow=false, pdfstartview={FitH}, pdfcreator={Daniel Larremore}, pdfproducer={Daniel Larremore}, pdfkeywords={} {} {}, pdfnewwindow=true, colorlinks=true, linkcolor=red, citecolor=Green, filecolor=magenta, urlcolor=cyan,}

\usepackage{enumitem}

\newcommand{\dx}[0]{\displaystyle\frac{d}{dx}}
\newcommand{\dy}[0]{\displaystyle\frac{dy}{dt}}
\newcommand{\so}[1]{\textcolor{red}{#1}}

\usepackage{parskip}

\date{}
\begin{document}

%%%%%%%%%% Authors
\author{CSCI 2897 - Calculating Biological Quantities - Larremore - Spring 2021}
%%%%%%%%%% Title
\title{Homework 3}
%%%%%%%%%% Abstract
\maketitle
%%%%%%%%%% Content

    %%%    
    %%%   
  %%%%%%%
   %%%%%
    %%%
     %
{\bf Notes:} Remember to (1) familiarize yourself with the collaboration policies posted on the Syllabus, and (2) turn in your homework to Canvas as a {\bf single PDF}. Hand-writing some or most of your solutions is fine, but be sure to scan and PDF everything into a single document. Unsure how? Ask on Slack! 

\section*{Hamstring curls}

{\bf Compute the following \so{and please show your work}, supposing that 
$a=
\begin{pmatrix}
1 \\ 
3 \\
2
\end{pmatrix}$,
$b=
\begin{pmatrix}
1 \\ 
4 \\
3
\end{pmatrix}$,
$c=
\begin{pmatrix}
0 \\ 
1 \\
1
\end{pmatrix}$. 
} 

\begin{enumerate}
	\item $a+b+c = $
	\item $a^{T}b = $
	\item $a+2b+3c = $
	\item $a b^{T} = $
\end{enumerate}

\section*{Calf raises} 

{\bf Using the same $a$, $b$, and $c$ as above, and with 
$D=
\begin{pmatrix}
1 & 0 &0  \\ 
0 & 1 & 2\\
1 & -1& 1
\end{pmatrix}$, solve the following, or explain why they cannot be solved. \footnote{Note: just like with regular multiplication, squaring a matrix means multiplying the matrix by itself!} Again, \so{please show your work}.} 

\begin{enumerate}[resume]
	\item $Da + c =$
	\item $a^{T}D + c = $
	\item $D^2=$
	\item $D^9 a - D^9b + D^9c = $
	\item $(a^T b) c = $
	\item $a^T (b c) = $
\end{enumerate}

\clearpage
\section*{Reasoning about matrices} 

\begin{enumerate}[resume]
	\item Suppose you know that $A$ is a symmetric $n \times n$ matrix. Let the matrix $B = A-A^{T}$. Let $x$ be a $n \times 1$ vector. Let $y=Bx$.  What is the {\bf dimension} of $y$? What is $y$?
	\item Let matrix $P$ have dimensions $5 \times 2$, $Q$ have dimensions $3 \times 2$, and $R$ have dimensions $3 \times 5$.  What is the dimension of $RPQ^{T}$? What about $QQ^{T}R$?
	\item In reference to the previous problem, write out {\it three} different ways that you could multiply the matrices $P$, $Q$, $R$ or their transposes to creates a $5 \times 5$ matrix. You may use each matrix at most two times in each multiplication.
	\item Suppose that I hand you two square $n\times n$ matrices, $X$ and $Y$. You multiply them and find that $XY=I$, where $I$ is the identity matrix. How and $X$ and $Y$ related?
	\item Compute the {\it trace} of $(XYX)(YX)(YXYXY)$.
\end{enumerate}

\clearpage
\subsection*{Computing with matrices}

\begin{enumerate}[resume]
	\item Let $A=
\begin{pmatrix}
2 & 1   \\ 
0 & 1 \\
\end{pmatrix}$
and $b=
\begin{pmatrix}
9 \\ 
5 \\
\end{pmatrix}$. Knowing that $Ax=b$, solve for the vector $x$.
\item What is the determinant of $A$?
\item Let $M=
\begin{pmatrix}
-2 & 1   \\ 
\alpha & -1 \\
\end{pmatrix}$. Knowing that $Mx=d$, where $d=\begin{pmatrix}
1 \\ 
1 \\
\end{pmatrix}$, solve for the vector $x$.  
\item The solution to the previous question fails to exist at a particular value of $\alpha$. What is this value? Explain why the solution ceases to exist at that value.
\item Extra credit: Make up a $2 \times 2$ matrix in Python, call it $L$. Make up a $2 \times 1$ vector in Python too, and call it $x$. Now do the following: (i) Compute $y=Lx$. (ii) Compute $x = y / || y ||$, where $||y||$ is the same as \texttt{numpy.linalg.norm(y)}. (iii) Repeatedly do (i) and (ii) 100 times by using x to compute $y$ and then using $y$ to get a new $x$, over and over. What do you notice? Start with a few new initial $x$  vectors and repeat this process. What do you find?  Explore changing $L$ and $x$ and write up some of your findings in under one page with some examples. 
\end{enumerate}

     %
    %%%
   %%%%%
  %%%%%%%
    %%%
    %%%

\end{document}